% Options for packages loaded elsewhere
\PassOptionsToPackage{unicode}{hyperref}
\PassOptionsToPackage{hyphens}{url}
%
\documentclass[
]{article}
\usepackage{amsmath,amssymb}
\usepackage{iftex}
\ifPDFTeX
  \usepackage[T1]{fontenc}
  \usepackage[utf8]{inputenc}
  \usepackage{textcomp} % provide euro and other symbols
\else % if luatex or xetex
  \usepackage{unicode-math} % this also loads fontspec
  \defaultfontfeatures{Scale=MatchLowercase}
  \defaultfontfeatures[\rmfamily]{Ligatures=TeX,Scale=1}
\fi
\usepackage{lmodern}
\ifPDFTeX\else
  % xetex/luatex font selection
\fi
% Use upquote if available, for straight quotes in verbatim environments
\IfFileExists{upquote.sty}{\usepackage{upquote}}{}
\IfFileExists{microtype.sty}{% use microtype if available
  \usepackage[]{microtype}
  \UseMicrotypeSet[protrusion]{basicmath} % disable protrusion for tt fonts
}{}
\makeatletter
\@ifundefined{KOMAClassName}{% if non-KOMA class
  \IfFileExists{parskip.sty}{%
    \usepackage{parskip}
  }{% else
    \setlength{\parindent}{0pt}
    \setlength{\parskip}{6pt plus 2pt minus 1pt}}
}{% if KOMA class
  \KOMAoptions{parskip=half}}
\makeatother
\usepackage{xcolor}
\usepackage{graphicx}
\makeatletter
\def\maxwidth{\ifdim\Gin@nat@width>\linewidth\linewidth\else\Gin@nat@width\fi}
\def\maxheight{\ifdim\Gin@nat@height>\textheight\textheight\else\Gin@nat@height\fi}
\makeatother
% Scale images if necessary, so that they will not overflow the page
% margins by default, and it is still possible to overwrite the defaults
% using explicit options in \includegraphics[width, height, ...]{}
\setkeys{Gin}{width=\maxwidth,height=\maxheight,keepaspectratio}
% Set default figure placement to htbp
\makeatletter
\def\fps@figure{htbp}
\makeatother
\setlength{\emergencystretch}{3em} % prevent overfull lines
\providecommand{\tightlist}{%
  \setlength{\itemsep}{0pt}\setlength{\parskip}{0pt}}
\setcounter{secnumdepth}{-\maxdimen} % remove section numbering
\ifLuaTeX
  \usepackage{selnolig}  % disable illegal ligatures
\fi
\usepackage{bookmark}
\IfFileExists{xurl.sty}{\usepackage{xurl}}{} % add URL line breaks if available
\urlstyle{same}
\hypersetup{
  hidelinks,
  pdfcreator={LaTeX via pandoc}}

\author{}
\date{}

\begin{document}

\includegraphics[width=0.67431in,height=0.49306in]{图片1.png}
\includegraphics[width=0.63264in,height=0.57639in]{图片2.png}

\textbf{Dockerized computational pipeline for SMRT sequence data
analysis}

Student Name: Xu,Hang

Student ID: 223050032

Programme: Bioinformatics

Course ID: MBI6013

Supervisor: Sun,Hao

Supervisor Signature:

\textbf{Abstract}

Text

\textbf{\hfill\break
}

\textbf{Introduction}

SMRT (Single Molecule Real-Time) sequencing is a third-generation
sequencing technology developed by Pacific Biosciences (PacBio). The
fundamental concept is that DNA polymerase serves as an exceptional
sequencing tool. Guided by the DNA template strand, it can synthesize a
new strand with remarkable accuracy, achieving an error rate of
10\textsuperscript{-5}, and the process is unbiased, eliminating issues
like GC bias. Furthermore, the enzyme reaction is not only swift,
reaching speeds of 750bp/s, but it is also efficient, requiring minimal
ATPs for energy and occupying a small volume. To observe the enzyme
reaction, we can utilize dNTPs with distinct fluorescent signals to
capture snapshots of the synthesis process. This approach allows us to
record the polymerization reaction of the enzyme in detail. Moreover,
the fluorescent signals are labeled on the polyphosphate moiety of
dNTPs, ensuring that they have no impact on the enzyme reaction.
Following the synthesis process, the fluorophores are cleaved from the
strand, eliminating any fluorophore presence within the strand. This
approach also prevents steric hindrance and background signal
accumulation, ensuring accurate and reliable observation of the enzyme
reaction.\textsuperscript{{[}1{]}}

SMRT is a high-throughput system, posing challenges in accurately
recording the reaction process of individual polymerases. To overcome
this, the ZMW Hole (Zero-Mode Waveguide Hole) is utilized to facilitate
the observation of a single DNA polymerase at a given time.
\textsuperscript{{[}2{]}}The physics behind this is the zero-mode
waveguide effect, a phenomenon where, if the aperture is smaller than
the wavelength of the laser light, the laser is unable to pass through
the aperture. Instead, it creates a tiny high-energy zone at the
aperture\textquotesingle s bottom, an area where the laser light can
reach. By positioning the polymerase within this confined space, we
create a microscopic environment conducive to observing the
polymerase\textquotesingle s polymerization reaction. The laser can
excite the fluorescence of dNTPs.

Compared to the existing second-generation sequencing (NGS)
technologies, SMRT sequencing offers several advantages. NGS sequencing
technologies typically have short library insert sizes of around 300
base pairs (bp), resulting in challenges related to capturing long-range
genomic information. Both library preparation and sequencing in NGS
require PCR amplification, resulting in uneven genome coverage and
difficulties in amplifying regions with high GC content. Consequently,
certain regions of the genome are challenging to sequence effectively
using NGS. Furthermore, NGS involves cluster generation, and errors that
occur during this process are amplified. The sequencing errors in NGS
are non-random, and increasing sequencing depth does not effectively
improve accuracy. As a result, NGS struggles to correct errors in
regions that are prone to frequent errors.

In contrast, SMRT sequencing addresses these limitations by allowing the
construction of libraries with longer fragments ranging from 1 to 100
kilobases (kb). This is achieved by attaching hairpin adapters to both
ends of the DNA double strand, forming a single-stranded circular
structure. The sequencing process occurs in a rolling-circle way, with
DNA polymerase initiating from the hairpin adapters within the zero-mode
waveguide (ZMW) holes. Moreover, SMRT sequencing eliminates the need for
PCR amplification during library preparation. This results in improved
library uniformity and eliminates GC bias, enhancing the accuracy and
representation of the entire genome. SMRT sequencing utilizes phi-29 DNA
polymerase instead of PCR polymerase, enabling a rolling-circle
replication mode. The random nature of errors introduced by the
polymerase during its activity can be reduced by increasing sequencing
depth, leading to higher accuracy. In fact, SMRT sequencing can achieve
accuracy levels up to Q50 or Q60 for shorter read lengths, surpassing
the accuracy attainable by NGS technologies.

Moreover, SMRT sequencing has the unique capability to detect DNA
modifications during the sequencing process. \textsuperscript{{[}3{]}}It
utilizes DNA polymerase kinetics to measure the fluorescence signals and
inter-pulse duration (IPD) between two consecutive bases. By analyzing
the time differences, it can identify potential base modifications.
Modified bases exhibit slower kinetics due to steric hindrance,
resulting in longer time intervals between incorporation events. This
temporal information allows for the identification and characterization
of various types of base modifications.\textsuperscript{{[}4{]}}

HiFi Reads, a remarkable feature of SMRT sequencing are used to sequence
long fragments. These long reads, typically spanning 10 to 25 kilobases
(kb), provide single-molecular resolution, allowing for the direct
sequencing of DNA or RNA (cDNA) without the need for amplification. With
individual molecular accuracy exceeding Q20 (99\%), HiFi Reads deliver
highly reliable and precise sequence information. What sets them apart
is their unbiased nature, as HiFi Reads are obtained without DNA
amplification, minimizing biases associated with GC content and sequence
complexity. This unbiased approach ensures a comprehensive and
representative view of the sequenced genome or transcriptome. possesses
strong strand displacement and continuous synthesis capabilities,
allowing it to unwind and replicate complex DNA structures. It enables
isothermal DNA amplification in vitro without the need for thermal
cycling, resulting in a high yield of high-molecular-weight DNA. With
3\textquotesingle{} to 5\textquotesingle{} exonuclease proofreading
activity, it exhibits superior fidelity compared to most other DNA
polymerases. Rolling Circle Amplification (RCA) is a nucleic acid
detection method that emulates the natural circular DNA rolling circle
replication process. Using circular DNA as a template, RCA utilizes a
short DNA primer that extends and displaces the previously synthesized
strand, facilitated by Phi29 DNA polymerase. This process leads to the
generation of repeated long single-stranded DNA products. Additionally,
HiFi Reads harness the power of Circular Consensus Sequencing (CCS),
utilizing Phi29 DNA polymerasea (rolling-circle DNA synthesis method) to
generate multiple subreads during repetitive sequencing. By performing
error correction within the same molecule, these subreads collectively
produce exceptionally accurate consensus reads.
\textsuperscript{{[}5{]}}

SMRT sequencing offers the advantage of longer read lengths compared to
NGS, as mentioned previously. In comparison to ONT sequencing, SMRT
sequencing provides additional benefits. In the first place, SMRT
sequencing exhibits higher raw accuracy than ONT sequencing. This is
achieved through circular consensus sequencing (CCS), which generates
multiple passes of the same DNA molecule, allowing for error correction.
In contrast, ONT sequencing may have higher error rates due to the
nature of nanopore technology. Secondly, SMRT sequencing excels in
identifying DNA base modifications, such as DNA methylation, by directly
detecting the kinetic changes caused by modified bases. This capability
is particularly valuable for epigenetic studies and understanding gene
expression regulation. While ONT sequencing can also detect base
modifications, it may have limitations in accurately distinguishing
between different types of modifications.

SMRT sequencing data analysis holds profound applications across diverse
areas of genomics research.\textsuperscript{{[}6{]}} In whole-genome
sequencing (WGS), its long read lengths and high accuracy enable the
comprehensive analysis of genetic variation, encompassing single
nucleotide polymorphisms (SNPs), insertions/deletions (Indels), and
structural variants. This comprehensive approach is invaluable in
studying human genetic diseases, cancer genomics, and population
genetics. Furthermore, SMRT sequencing\textquotesingle s application in
RNA sequencing contributes significantly to our understanding of dynamic
gene expression processes. By directly measuring the full length of
transcripts, it reveals the complexity of transcript isoforms,
alternative splicing variants, and non-coding RNAs, crucial for gene
regulation studies, the identification of novel functional transcripts,
and disease mechanism investigations. In DNA methylation research, SMRT
sequencing stands out for its unique ability to directly detect dynamic
changes caused by modified bases, enabling high-sensitivity
identification of DNA methylation sites. This is paramount in
epigenetics studies, genome stability assessments, and understanding the
mechanisms underlying disease occurrence and development. Moreover, SMRT
sequencing\textquotesingle s ability to differentiate between different
types of DNA methylation modifications, such as 5-methylcytosine (5mC)
and 6-methyladenine (6mA), offers powerful tools for delving into the
functional roles of these modifications in gene regulation.

Especially,SMRT sequencing demonstrates significant advantages in
analyzing the methylation of circulating free DNA (cfDNA) in plasma,
particularly for non-invasive early cancer screening.
\textsuperscript{{[}7{]}}While traditional NGS techniques face
challenges in detecting long cfDNA fragments, SMRT sequencing overcomes
this obstacle with its unique capabilities. Firstly, the ultra-long read
lengths of SMRT sequencing allow for the coverage of longer DNA
fragments, enabling more comprehensive capture of methylation
information. In contrast, traditional NGS techniques often struggle with
long cfDNA fragments, potentially leading to the omission of important
methylation information. SMRT sequencing, on the other hand, can
accurately reveal changes in methylation patterns, which often serve as
early signals for cancer occurrence and development. Secondly, SMRT
sequencing offers high accuracy and sensitivity, precisely identifying
methylation sites and their alterations. This precision is crucial for
distinguishing normal cells from cancer cells, assessing tumor burden
and metastasis. In comparison, NGS technologies may be limited in
accurately identifying methylation sites when detecting long cfDNA
fragments due to read length constraints. SMRT sequencing provides more
reliable methylation analysis results, offering a more accurate basis
for personalized treatment strategies. Additionally, SMRT sequencing has
the ability to directly detect methylation without the need for
additional chemical processing or conversion steps. This simplifies the
experimental workflow and reduces potential errors and biases. In
contrast, NGS techniques may require complex preprocessing and
conversion processes when detecting methylation, increasing experimental
complexity and potential sources of error.

Docker is an open-source containerization platform widely used in
cluster computing environments. It provides a lightweight, portable, and
scalable containerization solution that allows developers to package
applications and their dependencies into independent containers,
enabling fast deployment and portability.\textsuperscript{{[}8{]}}

In cluster computing, Docker offers several advantages. Firstly, Docker
containers provide isolated runtime environments, allowing different
applications to run independently in the same cluster without
interference. This isolation enhances deployment flexibility and reduces
failures and instability caused by conflicts between applications.
Secondly, Docker containers are lightweight and have fast startup times.
As containers share resources with the operating system kernel, they
generally have much faster startup times compared to traditional virtual
machines. This enables efficient dynamic scheduling and scaling of
containers in the cluster, better adapting to changes in computational
workloads. Additionally, Docker containers possess portability and
repeatability. Containers package applications and their dependencies
into a self-contained unit, simplifying and ensuring reliability in
application deployment and migration. Developers can create and test
containers in their local development environment and seamlessly deploy
them to any node in the cluster, without concerns about environment
configurations and dependencies. Lastly, the Docker ecosystem is rich
and active. Docker Hub provides a vast repository of public images,
allowing developers to easily access and share commonly used container
images. Moreover, Docker supports container orchestration tools such as
Docker Compose and Kubernetes, facilitating convenient and flexible
management and orchestration of large-scale containerized applications
in the cluster.\textsuperscript{{[}9{]}}

\textbf{\hfill\break
}

\textbf{Materials and Methods}

Text

\textbf{\hfill\break
}

\textbf{Results}

Text

\textbf{\hfill\break
}

\textbf{Discussion}

Text

\textbf{\hfill\break
}

\textbf{Conclusions}

Text

\textbf{\hfill\break
}

\textbf{References}

\begin{enumerate}
\def\labelenumi{\arabic{enumi}.}
\item
  Scheres, S. H. RELION: implementation of a Bayesian approach to
  cryo-EM 1048 structure determination. \emph{J Struct Biol}
  \textbf{180}, 519-530 (2012).
\end{enumerate}

\end{document}
